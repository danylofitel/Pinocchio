\documentclass[12pt]{report}

\usepackage{amsmath,amssymb,amsfonts,amsthm}
\usepackage{graphicx}
\usepackage{psfrag}
\usepackage{srcltx}
\usepackage[T2A]{fontenc}
\usepackage[utf8]{inputenc}
\usepackage[english]{babel}
\usepackage{fancyhdr}

\fancypagestyle{plain}{}
\pagestyle{fancy}
\fancyhead{}
\fancyfoot{}
\renewcommand{\headrulewidth}{0 pt}
\rhead{\thepage}
\setlength{\headwidth}{483 pt}
\setlength{\headheight}{0.8 pt}

\paperheight 297mm \paperwidth 220mm \oddsidemargin -7mm
\topmargin 0mm \textwidth 170mm \textheight 220mm \marginparsep
0mm \marginparwidth 0mm \headheight 0mm \headsep 0mm \footskip 5mm
\parskip 2pt

\DeclareMathOperator{\Bool}{Bool}
\newcommand{\Wr}{\mathop{\font\xx=cmsy10 at 18pt \hbox{\xx \char111}}}
\def\thnumbering{chapter}
\def\equa{\Leftrightarrow}
\def\x{{\mathcal{X}}}
\def\b{{\mathcal{B}}}

\newcommand{\bb}[1]{{\bf #1}}
\newcommand{\wsp}{\hspace{6pt}}
\newcommand{\re}{\mathbb{R}}

\newtheorem{theorem}{Theorem}[section]
\newtheorem{definition}{Definition}[section]

\makeatletter
\renewcommand{\subsection}{\@startsection{subsection}{2}{0mm}{-\baselineskip}{-5pt}{\bf}}
\makeatother


\begin{document}


\renewcommand{\bibname}{References}
\setcounter{tocdepth}{1}

\setcounter{page}{2}

\large

\thispagestyle{empty}
\tableofcontents
\thispagestyle{empty}


\chapter*{Introduction}
\addcontentsline{toc}{chapter}{Introduction}

True loneliness is when you don’t even receive spam.

\newpage


\chapter{Statement of the Problem}

\section{Scope and Applications}

A social networking service (or SNS) is a platform to build social networks or social relations among people who share similar interests, activities, backgrounds or real-life connections (definition from Wikipedia). The ultimate purpose of any social networking service is fast and efficient exchange of information, often with intention to present it to the largest audience possible.

The vast majority of most popular social network services rely on text messages as the main form of exchange of information. Because of their openness, social networks can be extremely useful in spreading malicious messages across wide audiences, both via private (addressed to a particular individual) and public messages. Conceptually social network spam is no different from e-mail spam as private messaging services of popular social networks are equivalent in their functionality to e-mail. Hence, we can generalize the problem of classifying spam messages for both these cases.

In general, the problem of spam detection depends heavily on the application's domain and can benefit from additional metadata available along with the text message. For example, in case of e-mails the mail header is the source of metadata. Modern spam filtering systems detect the vast majority of malicious mails by simply checking the sender's reputation before proceeding with analysis of the message body.

This, of course, applies to all messaging services. Maximum effectiveness can not be achieved without using all available data in addition to the message text. However, in most cases text analysis is the second stage preceded by a domain-specific filter. Therefore, we can further focus on statistical classification of spam for text messages without specific constraints.

Let us denote the set of all messages by $M$, and let $S \subseteq M$ be the set of spam messages and $L = M \setminus S$ be the set of legit messages. The ultimate goal is to obtain a decision function $f : M \rightarrow \{S, L\}$ that would determine whether a given message $m \in M$ is spam $(m \in S)$ or legitimate mail $(m \in L)$.

We shall look for this function by training a number of machine learning algorithms on a set of pre-classified messages $\{(m_1, c_1), (m_2, c_2), ..., (m_n, c_n)\}, m_i \in M, c_i \in \{S, L\}, 1 \leq i \leq n$. There are two aspects for the case of text messages: we have to extract features from text strings and we may have strict requirements for the precision of classifier.

\newpage

\section{Feature Selection}

The entities we need to classify are text messages that are given in the form of strings. Raw strings are not convenient objects to handle in this case. Most machine learning algorithms can only classify numerical objects or otherwise require a distance metric or other measure of similarity between the objects.

Before proceeding with machine learning we have to convert all messages to numerical vectors, so called feature vectors, and then classify these vectors. The simplest example of a feature vector is the vector of the numbers of occurrences of certain words in a message. 

Extraction of features usually means that some information from the original message is lost. On the other hand, the way feature vector is chosen is crucial for the performance of the filter. If the features are chosen so that there may exist a spam message and a legitimate mail with the same feature vector, then any machine learning algorithm will make mistakes no matter how good it is. A wise choice of features will make classification much easier while also fast. In most practical applications the most basic vector of word frequencies or its modification is used.

Note that at the stage of feature selection it is possible to include the features from the available metadata along with features from message text. In practice, however, it is much more important what features are chosen for classification than what classification algorithm is used.

Now let us consider those machine learning algorithms that require distance metric or scalar product to be defined on the set of messages. There does exist a suitable metric (edit distance), and there is a nice scalar product defined purely for strings [2], but the complexity of the calculation of these functions is sometimes too restrictive to use them in practice. So in this work we shall simply extract the feature vectors and use the distance/scalar product of these vectors.

\newpage


\chapter{Basic Methods}

\section{Naive Bayes Classifier}

Words.

\newpage

\section{k Nearest Neighbors Classifier}

Words.

\newpage

\section{Artificial Neural Network Classifier}

Words.

\newpage

\section{SVM Classifier}

Words.

\newpage

\section{Boosting}

Words.

\newpage


\chapter{Advanced Approaches}

\section{Combined Boosting}

Words.

\newpage

\section{Adaptive Feature Selection}

Words.

\newpage


\chapter*{Conclusion}
\addcontentsline{toc}{chapter}{Conclusion}

Words.

\newpage


\addcontentsline{toc}{chapter}{References}

\begin{thebibliography}{30}

\bibitem{Tretyakov} Konstantin Tretyakov. Machine Learning Techniques in Spam Filtering. Institute of Computer Science, University of Tartu. Data Mining Problem-oriented Seminar, MTAT.03.177, May 2004, pp. 60-79.

\bibitem{Carreras} Xavier Carreras, Lluis Marquez. Boosting Trees for AntiSpam
Email Filtering. TALP Research Center, LSI Department, Universitat Politecnica de Catalunya.

\end{thebibliography}


\end{document}